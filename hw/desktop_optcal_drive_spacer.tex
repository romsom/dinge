\documentclass{standalone}

%\input{common.tex}
\usepackage{tikz}
\usetikzlibrary{calc,intersections,through,backgrounds}
\tikzset{
  cut/.style={draw=red,line width=0.1mm},
  mark/.style={draw=green,line width=0.1mm},
  ignore/.style={draw=blue,line width=0.1mm},
}

\begin{document}
\noindent\begin{tikzpicture}[node distance=1cm]
  \coordinate (origin) at (0,0);
  \foreach \height/\width in {23.5mm/110mm} {
    % \path let in node[rectangle,draw=black,line width=0.1mm,minimum height = \height, minimum width=\width] (outline) at (\width / 2, -\height / 2) {};
    \coordinate (se) at (\width, -\height);
    \coordinate (sw) at (0, -\height);
    \coordinate (nw) at (0, 0);
    \coordinate (ne) at (\width, 0);
    \draw[cut] (ne) -- (nw);
    \draw[cut] (nw) -- (sw);
    \draw[cut] (sw) -- (se);
    \draw[cut] (se) -- (ne);

    \foreach \xoffset/\yoffset in {79mm/-11.5mm} {
      % led holes
      \path let in coordinate (hole_origin) at (13mm, -\height + 6mm - \yoffset);
      \foreach \xindex in {0, 1} {
        \foreach \yindex in {0, 1} {
          \path let \p1=(hole_origin) in coordinate (hole_position) at (\x1 + \xoffset * \xindex, \y1 + \yoffset * \yindex) {};
          \draw[cut] (hole_position) circle (1.5mm);
        }
      }
    }
  }
\end{tikzpicture}
\end{document}